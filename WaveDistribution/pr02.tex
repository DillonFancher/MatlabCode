
\documentclass[11pt]{article}

%------------------------Dimensions----------------------------

\topmargin=0.0in
\oddsidemargin=0.0in           % 1in margins at left and right
\evensidemargin=0in
\textwidth=6.5in               % US paper is 8.5in wide
\marginparwidth=0.5in

\headheight=0pt                % 1in margins at top and bottom
\headsep=0pt
\topmargin=0in
\textheight=9.0in              % US paper is 11.0in high

% adjustments...
\addtolength{\topmargin}{-0.5in}
\addtolength{\textheight}{1.0in}
\addtolength{\textwidth}{1.0in}
\addtolength{\oddsidemargin}{-0.5in}
\addtolength{\evensidemargin}{-0.5in}
            
\pagestyle{empty}             % don't put page numbers at the bottom

%------------------------Commands----------------------------
            
\newcommand{\ds}{\displaystyle}
\newcommand{\be}{\begin{enumerate}}
\newcommand{\ee}{\end{enumerate}}
\newcommand{\bi}{\begin{itemize}}
\newcommand{\ei}{\end{itemize}}

\newcommand{\mA}{\mathbf{A}}
\newcommand{\mb}{\mathbf{b}}
\newcommand{\mx}{\mathbf{x}}
\newcommand{\my}{\mathbf{y}}
\newcommand{\mr}{\mathbf{r}}

             
%*********************************************************************

\begin{document}
\baselineskip=14pt

\centerline{APPM 3050 \hfill Project \#02 \hfill Spring 2013}
\centerline{ Due Monday, May 6, 2012}
\bigskip
\bigskip
%*********************************************************************

This project examines the numerical solution of the wave equation
in a liquid in
one and two dimensions, subject to various boundary conditions.
Ultimately you will account for variable depth and see how it changes
the propagation of the waves.

%*********************************************************************
\section{1-D Wave equation}
If one defines $u(x,t)$ to be the vertical displacement of a
wire (or a fluid surface),
measured from an equilibrium position of zero, then
the application of Newton's law (sum of the forces equals mass
times acceleration) leads to the basic one dimensional
wave equation
\begin{equation}
\frac{1}{c^2}\frac{\partial^2 u}{\partial t^2}
     =  \frac{\partial^2 u}{\partial x^2} \,,          \label{1Dwave}
\end{equation}
where $c$ is the propagation speed of the wave, $x$ is the horizontal
distance measured from the left side of the wire, and $t$ is time.
To solve this we require two initial conditions and two boundary
conditions. For initial conditions, we will assume an initial
displacement of the wire specified by the function $f(x)$,
\begin{equation}
u(x,0) = f(x) \,,                \label{initDisplacement}
\end{equation}
and an initial speed of zero,
\begin{equation}
u_t(x,0) = 0 \,.                \label{initSpeed}
\end{equation}
For boundary conditions we will use a known but variable displacement
for the left side of the wire,
\begin{equation}
u(0,t) = g(t) \,,                \label{leftSide}
\end{equation}
and a rather flexible form for the boundary condition on the right
side of the wire
\begin{equation}
\alpha u(L,t) + \beta \frac{du(L,t)}{dx} = \gamma \,.      \label{rightSide}
\end{equation}
By selecting different values for $\alpha$, $\beta$ and $\gamma$,
one can form several different physically realistic boundary
conditions. What remains, is to consider the calculations and results that
follow for specific functions $f(x)$ and $g(t)$, and particular
values of $\alpha$, $\beta$ and $\gamma$.

%*********************************************************************
% \vfill
% \hrule
% \newpage
%*********************************************************************

\subsection{Fixed boundary conditions at both ends}
During some of the recent labs
you wrote a Matlab script to numerically calculate
$u(x,t)$ for a ``plucked'' wire by solving (\ref{1Dwave})
with specific initial displacement functions $f(x)$, and
zero displacement at the left and right sides of the wire, $g(t) = 0$.
Using the shorthand notation $u(x_i, t_k) = u_i^k$
we used a simple finite difference approximation for each of the
second derivative terms in (\ref{1Dwave}) to arrive at
\begin{equation}
( \, u_i^{k+1} - 2u_i^k + u_i^{k-1} \, ) / \Delta t^2
   = c^2 ( \, u_{i-1}^k - 2u_i^k + u_{i+1}^k \, ) / \Delta x^2 \, ,
                                                         \label{wave}
\end{equation}
from which one could solve for
\begin{equation}
u_i^{k+1} = \lambda u_{i-1}^k + 2(1-\lambda) u_i^k + \lambda u_{i+1}^k
             - u_i^{k-1}  \,,    \label{normal}
\end{equation}
where $\lambda = ( c \Delta t / \Delta x )^2$.
In this expression, it was assumed that $u$ values at times $t_k$ and
$t_{k-1}$ are known. Hence (\ref{normal}) is valid for $k = 2, 3, 4, \ldots$.
There is, however a problem when trying to calculate $u_i^2$ values
using (\ref{normal}) since although $u_i^1 = f(x_i)$ from the
initial condition,
there are no $u_i^0$ values. To solve this problem, we invoked the
initial condition (\ref{initSpeed}) and created phantom values $u_i^0$.
Then, by coupling the finite difference forms of (\ref{initSpeed})
and (\ref{wave}) we arrived at the special case
\begin{equation}
u_i^2 = ( \, \lambda u_{i-1}^1 + 2(1-\lambda) u_i^1 + \lambda u_{i+1}^1
             \,) / 2 \,.    \label{initial}
\end{equation}

Thus to summarize, we get $u_i^1$ values from the initial condition
$u(x,0) = f(x)$,
we get $u_i^2$ values from (\ref{initial}), and finally we get the
remaining $u_i^{k+1}$ values from (\ref{normal}) for $k = 2, 3, 4, \ldots$. 

Included with this document is a file named {\tt WaveOrig.m} that
performs the above calculations. You should probably use
this code to examine the vibrating wire for several
initial displacements, such as $f(x) = x ( 1-x )^3$, and
$f(x) = x ( 1-x^3 )$, just to get a feel for how waves behave. 



%*********************************************************************

\subsection{Left side boundary moves}
In lab you further modified your wave equation script to examine
the wave propagation resulting from a wiggle at the left end of an
initially stationary wire at an equilibrium position. Specifically,
with a zero initial displacement of the wire ($u(x,0) = f(x) = 0$),
you wiggled the left end (a non-zero $g(t)$ in (\ref{leftSide}))
and watched the wave travel down the wire and reflect back from the
right--hand side of the wire.

The file called {\tt Wave1D.m} performs these calculations.
Note that it calls a file named {\tt leftBoundary.m} that contains
the definition of $g(t)$.



%*********************************************************************

\subsection{Flexible boundary condition on right}
Finally, in a recent lab, you modified your wave program to allow
for the boundary condition
$\ds \alpha u + \beta \frac{d u}{d x}= \gamma$
at the right--hand side of our wave region where $x = L$.
By selecting different values of $\alpha$, $\beta$, and $\gamma$
you can change the right boundary from a fixed value to
a free-floating condition. The finite difference version
of our new right--hand side boundary condition can be used to
solve for, and eliminate, any phantom values of $u$ at the
$x=L$ location.

Included with this document is a file called {\tt Wave1DFlex.m} that
performs these calculations. Note that it calls an additional file
named {\tt rightBoundary.m} that determines the value of the phantom
node at the right side of the wire.



%*********************************************************************

\subsection{Variable depth model}
Now we'll examine the effect of variable depth on the propagation
of our 1-D wave. If one assumes the depth of the fluid measured down from
the equilibrium surface of the fluid is $h(x)$, then the wave equation
(\ref{1Dwave}) must be modified. The result is
\begin{equation}
\frac{1}{c^2}\frac{\partial^2 u}{\partial t^2}
       = \frac{\partial}{\partial x} \left( h(x)
             \frac{\partial u}{\partial x} \right) 
       = \frac{dh}{dx} \frac{\partial u}{\partial x}
       + h(x) \frac{\partial^2 u}{\partial x^2} \,,
                                                      \label{1DwaveDepth}
\end{equation}
Clearly, if $h(x) = 1$, then (\ref{1DwaveDepth}) becomes (\ref{1Dwave}).
If one creates the finite difference version of (\ref{1DwaveDepth}),
you will notice that it essentially has the same from as
(\ref{normal}) and (\ref{initial}) but with different coefficients in front
of the three $u^k$ terms.

\begin{itemize}
\item Make a copy of {\tt Wave1DFlex.m} and call it {\tt Tsunami1D.m}.
\item Modify this file to incorporate the finite difference version
      of (\ref{1DwaveDepth}). It should call a supporting file named
      {\tt depth1D.m} that was included in the package of files. Notice
      that {\tt depth1D.m} accepts a vector of $x$ locations and returns
      two vectors of $h(x)$ and $dh/dx$ values.
\item Make a nice plot of $u$ versus $x$, as time increases.
\end{itemize}


%*********************************************************************
% \newpage

\section{2-D Wave equation}
Now we'll examine the propagation
of a 2-D wave confined to a rectangular region of dimension
$L_x$ by $L_y$. Since the surface of the liquid now
varies with both $x$ and $y$ location and with time $t$,
we will refer to $u(x_i, y_j, t_k) = u_{i,j}^k$.
If one assumes the depth of the fluid measured down from
the equilibrium surface position is now $h(x,y)$, then equation
(\ref{1DwaveDepth}) must be modified. The result is
\begin{equation}
\frac{1}{c^2}\frac{\partial^2 u}{\partial t^2}
                    = \frac{\partial}{\partial x} \left( h(x,y)
                      \frac{\partial u}{\partial x} \right)
                    + \frac{\partial}{\partial y} \left( h(x,y)
                      \frac{\partial u}{\partial y} \right)
                                                  \,.    \label{2DwaveDepth}
\end{equation}

The two initial conditions will be a flat surface at equilibrium,
$\ds u(x,y,0) = 0$, and zero velocity, $u_t(x,y,0) = 0$.
To simplify your code, assume free-floating
conditions on three out of the
four sides of the rectangular region in the $x$-$y$ plane. Specifically,
$u_x(L,y,t) = 0$, $u_y(x,0,t) = 0$, $u_y(x,L_y,t) = 0$. The remaining
boundary condition at $x=0$ is, again, a short sinusoidal wiggle.
Specifically $u(0,y,t) = \sin(2\pi t)$ for
$0 \le t \le 0.5$, and $u(0,y,t) = 0$ for $0.5 \le t$.
You may also assume that $\Delta x = \Delta y$.

\subsection{2-D uniform depth}
To start the 2-D analysis, take a look at the simple case of
uniform depth, $h(x,y) = 1$.
In this case, the general 2-D wave equation (\ref{2DwaveDepth}) becomes
\begin{equation}
\frac{1}{c^2}\frac{\partial^2 u}{\partial t^2}
                    = \frac{\partial^2 u}{\partial x^2}
                    + \frac{\partial^2 u}{\partial y^2}
                                                  \,.    \label{2DwaveUniform}
\end{equation}

\begin{itemize}

\item  Convert (\ref{2DwaveUniform}) to finite difference form.
The result should involve values of $u$ at five spatial locations
in the $t_k$ plane. It should
also involve one value of $u$ in the $t_{k-1}$ plane,
one value in the $t_{k}$ plane,
and one value in the $t_{k+1}$ plane, all at location $(x_i, y_j)$.

\item Make a copy of {\tt Wave1DFlex.m} and call it {\tt  Wave2D.m}.
Modify this file to incorporate the finite difference version of
(\ref{2DwaveUniform}).

\item You may simplify your {\tt  Wave2D.m} code by assuming
free-floating conditions on the top, bottom and right boundaries
of the rectangular region. In other words, $u_n = 0$ where $n$
is $x$ on the right side, and $n$ is $y$ on the top and bottom.
This allows you to eliminate the ``flexible'' boundary condition
(\ref{rightSide}) and replace it with the much simpler condition
$u_n = 0$ on the top, bottom, and right sides.

\item Make a nice illuminated surface plot of the
fluid surface over the rectangular region.
You might want to use a {\tt `cool'} colormap.

\end{itemize}


\subsection{2-D variable depth}
We must finally address the most general problem of solving
(\ref{2DwaveDepth}) on a rectangular region with the boundary
and initial conditions from the previous section.

\begin{itemize}

\item  Convert (\ref{2DwaveDepth}) to finite difference form.
The result should still involve five values of $u$ in the $t_k$ plane.
It should also still involve one value of $u$ in the $t_{k-1}$ plane,
one value in the $t_{k}$ plane,
and one value in the $t_{k+1}$ plane, all at location $(x_i, y_j)$.
However, the coefficients on the various $u$ terms should now
involve local values of $h$, $dh/dx$ and $dh/dy$.

\item Make a copy of {\tt  Wave2D.m} and name it {\tt Tsunami2D.m}.
Modify this file to incorporate the finite difference version of
(\ref{2DwaveDepth}).

\item Your {\tt Tsunami2D.m} file should call a supporting file named
{\tt depth2D.m} that accepts two-dimensional arrays of $x$ and $y$
coordinates and returns three two-dimensional arrays of $h$,
$\partial h/\partial x$ and $\partial h/\partial y$ values at
specific $(x,y)$ locations.

Be careful to keep the bottom of your ``ocean'' below the water
surface. In other words, don't let $h(x)$ become negative at any
location in your $L_x$ by $L_y$ region.

\item You may still simplify your {\tt Tsunami2D.m} code by assuming
free-floating boundary conditions on the top, bottom and right sides
of the rectangular region. In other words, $u_n = 0$ where $n$
is $x$ on the right side, and $n$ is $y$ on the top and bottom.

\item Make a nice illuminated surface plot of the surface of the
fluid over the rectangular region.

\end{itemize}

%*********************************************************************


\section{The project write-up}
\begin{itemize}

\item You must work in groups of size two or three.
Even if more than three of you work together to write
your programs, you must break up into smaller groups
when it comes time to write the summary report.

\item The document should be done in LaTeX. I have included the
LateX file for this document to provide you with another sample
document done in LaTeX.

\item The style should be similar to that used in Calculus III
or Differential Equations. In short, it should look like a real
project report --- something that you might submit to an employer
or a journal to clearly explain what the problem was all about,
how you approached solving it, what your results are, and what
the solution and results mean.

\item You should submit a hard--copy of the report by the due
date listed at the top of this document. Hard--copies will
be accepted until 5 pm. Electronic files will be accepted
until 11:59 pm.

\item You should also submit all of your Matlab code, any
supporting Matlab files, and the LaTeX file for your report.
Please put the Matlab code in one folder and the LaTeX file in
another. Be sure to use the naming conventions at the end of
this document.

\end{itemize}

%*********************************************************************



\vfill
% \hrule
% \newpage
% \medskip

%*********************************************************************
\hrule
\medskip
\noindent
Electronic work should:
\begin{itemize}

\item have all code and any output inside a directory
      (folder) named {\tt YourLastNamePR02}. This directory should be ``zipped'' 
      in an archive file named {\tt YourLastNameHPR02.zip} and submitted
      to D2L by 11:59 PM on the due date. 
\item contain the following comment lines at the beginning of each file,
      (with the appropriate information filled in):
{\footnotesize\tt\begin{verbatim}
% A one-line description of the program
% Your name
% Today's date
% APPM 3050, Project #02
\end{verbatim}}

\item contain sufficient comments to clearly explain what is being
      done at various points in the program.

\end{itemize}

%*********************************************************************

\end{document}

